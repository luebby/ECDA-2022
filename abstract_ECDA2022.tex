\documentclass{svmult}
\usepackage{amsmath}
\usepackage{amsfonts}
\usepackage{latexsym}
\usepackage{graphicx}
\usepackage{multicol}
\usepackage[latin1]{inputenc}
\usepackage{lipsum} 
%\setlength\voffset{-3cm}
\setlength\textheight{26cm}


\begin{document}
\title*{Acting Data-Driven -- But How?}

\author{Karsten L\"{u}bke\inst{1}\and
Matthias Gehrke\inst{2}\and
J\"{o}rg Horst\inst{3}\and
Sebastian Sauer\inst{4} \and
Gero Szepannek\inst{5}}
\institute{FOM University of Applied Sciences \texttt{karsten.luebke@fom.de} 
\and FOM University of Applied Sciences \texttt{mathias.gehrke@fom.de}
\and Bielefeld University of Applied Sciences \texttt{joerg.horst@fh-bielefeld.de}
\and Ansbach University of Applied Sciences \texttt{sebastian.sauer@hs-ansbach.de}
\and Hochschule Stralsund - University of Applied Sciences \texttt{gero.szepannek@hochschule-stralsund.de}}

\maketitle

\begin{abstract}
In many cases, data is used to draw conclusions, e.g., to support decision-making processes.
But quite often, the data is inconclusive, with Simpson's paradox being the most prominent example where the adjusted or unadjusted effect may even show in opposite directions.
But as causal inference is one of the data science tasks (Hern\'{a}n et al., 2019), the qualitative assumptions about the data generating process need to be considered and discussed in order to draw correct conclusions.

In a simulated scenario we asked students as well as practioners which conclusions they draw from am given regression output. In the simulation the sign of the estimate of interest changes if a covariable is added to the model.
First the result is presented without a causal diagram, afterwards with a causal diagram. 
The results show that the chosen conclusions are quite often wrong given the information provided (with or without the causal diagram). 
As a consequence for data science projects, more emphasis should be put on the mapping and link between subject matter knowledge and data modeling to {\em avoid drowning in the data}.


\end{abstract}

\begin{thebibliography}{1}

\item[]
HERN\'{A}N, M.A and HSU, J. and  HEALY, B. (2019): {\em A Second Chance to Get
Causal Inference Right: A Classification of Data Science Tasks CHANCE, 32(1), 42--49}. 

\end{thebibliography}

\section*{Keywords}
DATA LITERACY, CAUSAL INFERENCE, DECISION MAKING, DATA SCIENCE EDUCATION 

\end{document}
